\section{Introduction}
Transaction is a useful programming abstraction for manipulating 
shared data. In transactional programming,  a set of  shared data 
operations that has been enclosed by begin() and end() transaction 
directives get executed as a one atomic unit. The consistency, 
isolation and durability guarantees of those operations
get handled by the runtime, executing the transactions. 

Modern databases/data-stores manage large amounts of data. 
Servicing data operations using a single node is not feasible 
due to lack of data capacity and compute resources. Modern large-scale 
datastore deployments solve this by partitioning the 
data store among multiple compute/data nodes. In such environment
transactions are distributed.

Transactional programming has been extensively studied in 
computer science literature over the years. However, recent
hardware advancements in networks (RDMA) and compute (multic-cores) and
storage/memory (NVM) may have made some of the well known bottlenecks
invalid -- thus encourage us to evaluate/re-design these mechanisms for
modern hardware/application parameters.

In this paper, we are going to look at the specific problem of creating
scalable timestamps for the purpose of distributed event ordering.
There has been several recent work on the topic~\cite{namdb, cicada}. They
either optimize timestamping for distributed transaction or node local 
transactions. However, most of the real world deployments falls in to
middle of that spectrum.

In this paper we introduce a scalable hybrid timestamping scheme 
that scales, both for node local and distributed transactions. 
We call it, loosely sychronized hybrid vector (LSHV) timestamps.
The LSHV timestamps combine fast hardware clock based 
node local timestamping with distributed cross node vector clocks.
The initial experiments using new clock yield nearly 40\% speedup
over baseline timestamping clocks.

In this work, we make the following contributions;
\begin{itemize}
	\item{\bf Loosely-sychronized hybrid vector clocks} - we use hardware clock based timestamps to 
		scale node local transactions and use conventional vector clocks to scale distributed
		transactions. We further show how to combine these two concepts to create a novel hybrid 
		timestamp technique.
	\item{\bf Implementation and evaluation} - we implement and evalute the proposed 
		timestamp mechanism on a real MVCC based key-value store.
\end{itemize}

