

\section{Related Work}
In this section we list down the most recent and important work that we 
found related to our work. 

\noindent{Multi-core transactions}
With widespread use of multi-core machines, and terabytes of main memory
it is possible to run complete data management system within a single 
physical node. Cicada~\cite{cicada} implments an in-memory, serializable
transaction processing database with multi-version concurrency control. They
introduce vector clocks based time-stamping for multi-cores. Echo~\cite{echo}
implements MVCC based key-value store for NVM based multi-core systems. Echo 
uses a global timestamp for versioning. GTEcho differs from these works as
our main goal is to scale time-stamping both within single node and across nodes.

\noindent{Distributed transactions}
FARM~\cite{farm} implements a distributed shared memory programming using distributed 
transactions. There system is heavily optimized for one-sided RDMA operations.
Fasst~\cite{fasst} also implements distributed RDMA based trasactions, but with 
optimized RDMA based RPC. Specifically they use unreliable datagram connection
for fast RPC using RDMA. Both of the above uses conventional 2PL based concurrency
control for distributed transactions. NAM-DB~\cite{namdb} implements OCC based 
distribtued transactions using RDMA primtives. They scale cross node timestamping
using per-thread based vector clocks. They do not optimize timestamping for 
partitioned data stores ( compute is always remote to the data).

\noindent{Hardware timestamps}
Spanner~\cite{spanner} implements strictly serializable distributed transactions
using synchronized hardware clocks and introduces Truetime abstractions for 
timestamp based ordering. GTecho uses hybrid timestamps where cross node operation
ordering and optimize local node operation ordering using commodity hardware clocks.

To the best of our knowledge GTEcho is the only system that uses hybrid vector
clock based timestamps to accelerate both local and distribtued transactions.
