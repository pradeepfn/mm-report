\section{Scalable Timestamping}
\begin{figure}[]   
	\centering
	\includegraphics[width=\linewidth]{figures/gtecho.pdf} 
	\caption{\bf GTEcho key-value store is partitioned across multi-core nodes. These nodes
	both hosts data and carry out compute (put/get operations) on clients behalf} 
	\label{fig:gtecho} 
\end{figure}

In this section we describe how we optimize timestamping overheads
associated with both node local and distributed trsansactions using combination
a of hardware and vector clocks. The targeted system architecture is shown in Fig~\ref{fig:gtecho}.
The key intuition is that, the node local transactions
can be ordered using hardware clocks available in modern CPUs while distributed 
transactions are ordered using per-thread vector clocks that eliminates RDMA atomic operations
on single memory location. Further, we show how to combine these
two timestamping techniques to form a new type of hybrid clock based timestamping. 
We name our new vector clock abstraction loosely Synchronized hybrid vector clocks ( LSHV clocks).

\subsection{Loosely Synchronized Hybrid Vector Clocks}

A timestamp using LSHV clock on node 1, that has $i$ hardware threads, can be represented using,
\[T_R = <ht_{1...i},t_{i+1}, t_{i+2}, ..., t_n>\]

Here each component in $T_R$ represent a unique, monotonically increasing counter -- thus a vector
clock. Each $t_i$ timestamp represents a counter used by remote transaction execution thread/worker.
The vector itself is stored in a one of the GTEcho partitions, -- thus each vector has a home partition.
Transaction execution threads belonging to home partition are treated special. They are combined together
to share a synchronized hardware clock $ht_{i...j}$ where $j - i + 1$ is the number of transaction execution
threads on the home partition. For an example, the $T_R$ given above localted on parition 1 of GTEcho
cluster and the partition has $i$ transaction execution threads.

\noindent{\bf Commit Timestamps}
We now explain how to use LSH vectors for commit timestamping. There are two distinct usecases based on the
locality of the transaction execution thread,

\noindent{\bf local to the LSHV}
If the transaction execution thread is local to the LSHV home partition, then $T_R[i] = ht_{i...j}$. 
That is, Execution thread uses hardware clock based timestamps. As the timestamping is node local the
operation only involves thread local instructions. However, instead of using atomic instructions to 
obtain thread safe timestamps we create hardware thread local timestamps that completely eliminate 
costly atomic operations. We differ the dicussion of safely creating hardware clock based timestamps
that can serve as ordering primitive till the next subsection.

\noindent{\bf remote to the LSHV}
If the transaction execution thread is remtoe to the LSHV home partition, then $T_R[i] = t_i $. The $t_i$
represents the latest commit timestamp that was made by the execution thread $i$. Thus incrementing the 
timestamp can be done without using a RDMA operation -- as this is the latest timestamp for this particular 
thread as per the vector clock. Therefore the commit timestamp would be $t_i + 1$

Hardware based clock timestamp component of the LSHV eliminates the need for costly atomic counters and
the vectorizing the remote timestamps remove the RDMA roundtrips while timestamping. During the commit,
as we describe in the basic SI protocol, the transaction thread has to validate the read/write set before
installing the new values. If the commit is successful, it then updates the 
$T_R$ to reflect the latest commit timestamp.

\noindent{\bf Read Timestamps}
Transaction threads read $T_R$ during the first transactional read from each of the partition and use it
as the read-timestamp to construct GTEcho snapshot while transaction execution. It uses them to querry
the hash-table to find visibile data versions for a given key.

\subsection{Ordering Local Updates}
Here we explain how we generate $ht_{i...j}$ timestamps for $T_R$.
We order local updates based on CPU clocks. Each CPU core has a mononotonically increasing clock. However
these clocks are not synchronized with each other -- we cannot derive total ordering of events based on hardware thread
local time. This is due to clock bootstrap time for each of the core differs due to electrical signaling delays during machine
startup. However, it has shown that these individual hardware clocks maintain a constant clock-skew with each other once they are properly bootstrapped. Therefore with conservative clock ordering boundary similar to Google Spanner's~\cite{spanner}
Truetime, we can use hardware clocks to order events -- specifically node local commit timestamping for transactions.

Recent work Ordo~\cite{ordo} has shown how to derive a conservative ordering boundary to order events based on hardware clocks.
They use software mechanism - spefically cacheline based messaging to derive a safe ordering boundary that enable comparison
of hardware clock timestamps generated by two distinct cores. We use the same technique to derive conservative ordering boundaries and
use the following API for hardware clock based timestamping.


\begin{itemize}
	\item {\bf get\_time()} - create a hardware clock based timestamp. Takes care of instruction ordering while doing so.
	\item {\bf new\_time(time\_t t)} - create a hardware timestamp, that is greater than the provided timestamp. Makes sure the returned 
		timestamp is safe to compare with ordering boundary offset.
	\item {\bf cmp\_time(time\_t t1, time\_t t2)} - compares two hardware timestamps. 
\end{itemize}



